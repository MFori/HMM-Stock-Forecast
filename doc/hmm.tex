\section{Skrytý markovský model}

Skrytý markovský model (HMM) je konečný automat s fixním počtem stavů, představen v 70. letech jako nástroj pro rozpoznávání řeči. Tento model založen na statistických metodách najde mnoho aplikací jako je rozpoznávání řeči, analýza sekvence DNA, rozpoznání ručně psaného textu a další.

\subsection{Definice}
HMM je charakterizován:
\begin{itemize}
    \item \(N\) - počet stavů modelu
    \item \(M\) - počet symbolů na stav
    \item \(T\) - délka pozorované sekvence
    \item \(O\) - pozorovaná sekvence \(O_1, O_2, ... O_T\)
    \item \(A = \{a_ij\}\) - pravděpodobnost přechodu ze stavu \(i\) do stavu \(j\)
    \item \(B = \{b_j(O_t)\}\) - pravděpodobnost pozorování \(O_t\) ve stavu \(j\)
    \item \(\pi = \{\pi_i\}\) - pravděpodobnost bytí ve stavu \(i\) v čase t=1
    \item \(\lambda = \{A, B, \pi\}\) - HMM model
\end{itemize}

Pokud jsou pravděpodobnosti spojitě rozděleny, máme spojitý HMM. V této práci předpokládáme, že pravděpodobnost pozorování je Normální (Gaussovo) rozdělení, potom \(b_i(O_t) = N(O_t = \mu_i, \sigma_i)\), kde \(\mu_i\) a \(\sigma_i\) jsou střední hodnota a rozptyl rozdělí pro stav \(S_i\), potom jsou parametry HMM definovány jako:

\[ \lambda = \{A, \mu, \sigma, \pi\} \]

\subsection{Problémy a jejich řešení}
Najdeme tři problémy při řešení problémů pomocí HMM:
\begin{enumerate}
    \item Pro pozorování \(O = \{O_t,t=1,2,...T\}\) a model \(\lambda = \{A,\mu,\sigma,\pi\}\) najít pravděpodobnost pozorování \(P(O|\lambda)\)
    \item Pro pozorování \(O = \{O_t,t=1,2,...T\}\) a model \(\lambda = \{A,\mu,\sigma,\pi\}\)
    najít sekvenci stavů, která nejlépe popisuje pozorování
    \item Pro pozorování \(O = \{O_t,t=1,2,...T\}\) zkalibrovat parametry modelu \(\lambda = \{A,\mu,\sigma,\pi\}\)
\end{enumerate}
\begin{enumerate}
    \item Pravděpodobnost pozorování najdeme pomocí algoritmu Forward nebo Backward
    \item Sekvenci stavů najdeme pomocí algoritmu Viterbi
    \item Model zkalibrujeme pomocí algoritmu Baum-Welch
\end{enumerate}

V naší práci využijeme algorimus Forward pro řešení prvního problému a algoritmus Baum-Welch pro řešení třetího problému (tento alogritmus využívá Forward a Backward). Druhý problém řešit nepotřebujeme.

\clearpage
