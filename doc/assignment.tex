\section{Zadání}

Cílem je podívat se na stávající literaturu a vyprat si pro vás zajímavý simulační problém, který zkusíte reimplementovat. V minulých letech se objevila řada věcí - celulární automat pro modelování šíření požáru, vzniku mraků nebo pohybu kapaliny v terénu, predikce cen akcií postavené na markovských modelech, simulace různého chování hmyzu a podobně. Není nutné přijít s vlastním, originálním řešením, může jít (a očekává se že půjde) o replikační studii, moc se jich nedělá.

Součástí zadání pak bude přehled state-of-the-art - co v dané oblasti už existuje, jaké nástroje je možné využít a podobně.

\subsection{Skrytý markovský model pro predikci cen akcií}

Cílem semestrální práce by měla být analýza odborné literatury na téma predikce cen akcií za pomoci skrytého markovského modelu (dále HMM) a jeho následná implementace.

Práce bude vycházet ze článku Hidden Markov Model for Stock Trading (Nguyen N., 2018) \cite{Nguyen}, který představuje aplikaci HMM pro obchodování s akciemi (jako příklad je použit index S\&P 500) na základě jejich predikce. Autor začíná použitím 4 kritérií pro odhad predikční chyby, aby určil optimální počet stavů pro HMM, vybraný čtyřstavový HMM implementovaný dle \cite{Hassan} je použit k predikci měsíčních uzavíracích cen indexu S\&P 500.

Výsledkem práce by měla implementace modelu v podobě desktopové aplikace, která dle historických dat (ať už poskytnuta uživatelem, či získána automaticky např. z finance.yahoo.com) natrénuje model a následně zobrazí v grafu jak trénovací data, tak predikovaná data.

V první části by měl mít implementovaný HMM fixní počet stavů, následně by mohl být rozšířen tak, aby dle kritérií pro odhad predikční chyby vybral vhodný počet stavů, který se může lišit dle konkrétního akciového titulu.

\clearpage
