\section{Zadání}

\par Velmi stručné - převod souboru ze zdrojového jazyka DBD do cílového jazyka DDL využitím principů formálních jazyků a překladačů.
\par Zadáním práce bylo vytvořit jednoduchou aplikaci, která ze zadaného DBD (Database definition) souboru vytvoří schéma databáze ve formátu DDL (Data definition language) pro některou z běžných relačních databází (MySQL, PostgreSQL, MariaDB...).

\par Semestrální práce vznikla zčásti pro předmět KIV/SAR a zčásti pro předmět KIV/FJP.

\par Pro převod aplikaci byly vzneseny zadavatelem následující požadavky:
\begin{itemize}
    \item využití některého z jazyků, které jdou pustit v prostředí z/OS (C, C++, Java, Fortran)
    \item snadné nasazení
    \item konzolové uživatelské rozhraní
    \item omezení zdrojového jazyka:
    \begin{itemize}
        \item omezení podporovaných \uv{mutací} DBD souboru na HDAM\footnote{Přesněji se jedná o typ databáze v prostředí IMS DB, který má specifické vlastnosti a je tím z uplně nejjednodušších}
        \item omezení rozeznatelných segmentů (základních prvků) DBD souboru na:
        \begin{itemize}
            \item DBD - databáze v relační db
            \item DATASET - vždy jen první dataset (nemá adekvátní náhradu v relační db)
            \item SEGMENT - table v relační db
            \item CHILD - column v relační db
            \item LCHILD - jen umět definovat, parsovat, ale netvořit cílový jazyk
        \end{itemize}
    \end{itemize}
\end{itemize}
\clearpage
