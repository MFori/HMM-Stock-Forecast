\section{Existující nástroje}

Pro použití HMM v pythonu (ve kterém bude aplikace implementována) existuje několik nástrojů, zde následuje seznam vybraných:

\begin{itemize}
    \item hmmlearn\footnote{https://github.com/hmmlearn/hmmlearn}
    \item pomegranate\footnote{https://github.com/jmschrei/pomegranate}
    \item PyHHMM\footnote{https://github.com/fmorenopino/HeterogeneousHMM}
\end{itemize}

\subsection{hmmlearn}
Knihovna pro práci s HMM v pythonu, prvotní verze programu byly testovány s touto knihovnou, ale celkem často padala.

\subsection{pomegranate}
Knihovna pro práci s HMM pythonu, narozdíl od hmmlearn fungovala spolehlivě a byla použita v prvotních fázích vývoje. Její integrace v programu zůstala a lze ji použit na místo našeho implementovaného modelu (viz uživatelská příručka). Je napsána v cythonu a obsahuje paralelizace, proto je mnohom výkonnější než naše implementace.

\subsection{PyHHMM}
Další knihovna pro práci s HMM v pythonu, s jednoduchou a transparentní implementací, některé její části byly použity i v naší implementaci.


\clearpage
