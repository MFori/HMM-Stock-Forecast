\section{Implementace}

Aplikace je implementována v pythonu, projekt je rozdělen takto:
\begin{itemize}
    \item \emph{data/data.py} - funkce pro načítání dat z csv souboru a finance.yahoo
    \item \emph{error/error.py} - funkce pro výpočet chyb viz výsledky
    \item \emph{hmm/criteria.py} - funkce pro výpočet kritérií viz výše
    \item \emph{hmm/hmm.py} - vlastní implementace hmm modelu
    \item \emph{plot/plot.py} - funkce pro vykreslení grafů
    \item \emph{utils} - pomocné funkce (parsování argumentů, logování)
    \item \emph{forecast.py} - funkce pro předpověď cen za využití hmm modelu
    \item \emph{main.py} - vstup programu
\end{itemize}

\subsection{HMM}
Implementace modelu vychází ze článku \cite{Nguyen}, části jsou inspirovány knihovnou \textbf{PyHHMM}.

\subsubsection{Rozhraní}
Model má stejné rozhraní jako knihovna \textbf{pomegranate} a proto je možná modely měnit viz uživatelská příručka. Rozhraní definuje dvě funkce:
\begin{lstlisting}
def fit(self, obs) -> None
def log_probability(self, obs) -> float
\end{lstlisting}
Funkce \emph{fit} slouží k natrénování modelu pomocí pozorované sekvence, funkce \emph{log\_probability} slouží ke zjištění pravděpodobnosti (jejímu logaritmu) pozorování.

\subsubsection{Forward algoritmus}
Pro zjištění pravděpodobnosti pozorování je využit algoritmus Forward, který je definovaný v článku \cite{Nguyen}. Je použita mírná modifikace (jako v knihovně \textbf{PyHHMM}), která počítá v logaritmickém měřítku.

\subsubsection{Backward algoritmus}
Dále je implementován algoritmus Backward, který pracuje na podobném principu jako předchozí a je použit u algoritmu Baum-Welch viz dále.

\subsubsection{Baum-Welch algoritmus}
Pro natrénování modelu je použit algoritmus Baum-Welch, který je popsán v článku \cite{Nguyen}.

\subsection{Predikce}
Predikce cen akcií je implementována přesně dle postupu popsaného v sekci 5.3.

\clearpage
